\documentclass[11pt,oneside]{report}
\usepackage[a4paper, margin=1.5in]{geometry}
%\usepackage{fullpage}
\usepackage{url}
\usepackage{graphicx}
\usepackage{harvard}
\usepackage{mathtools}
\usepackage{parskip}
\usepackage{subcaption}
\usepackage{bashful}

\citationmode{abbr}
\renewcommand{\baselinestretch}{1.5}

\title{An Open Source Computer Vision Library for Leapmotion}
\author{Daniel Hamilton 10026535,\\Computer Science,\\University of the West of England.}
	
\bash
texcount -sum -1 report.tex
\END

\begin{document}
	\maketitle
	\tableofcontents

	
	Word Count: \bashStdout.
	\begin{abstract}
	
	\end{abstract}	
	%----------------------------------------- Chapter 1 -----------------------------------------%
	\chapter{Introduction}
		\section{Computer Vision: An Intellectual Frontier}
			\subsection{What is Computer Vision?}		
				
				It is very hard for humans to know what their own biological vision really entails, and how difficult it is to reproduce on a computer.
				Information from the eyes is divided into multiple channels, each streaming different kinds of information to the brain.
				The brain then subconsciously groups and identifies the parts of the image to examine along with what parts to surpress.
				The way in which biological vision works is still largely unknown which makes it hard to emulate on computers \cite[p. xi]{book:multiViewGeo}.				
				Computer vision systems are still relatively naive, all they "see" is a grid of numbers.%from def:cv
				
				%By default there is no built in pattern recognition, or what some might call, intelligence.
				
				Computer vision is a vast field and hard to define.
				For the purpose of this report it will be defined as:
	
				\begin{quote}
					``\textit{The transformation of data from a still or video camera into either a decision or a new representation.
						All transformations are done for achieving some particular goal.}'' \cite[p. 2]{definition:cv}
				\end{quote}
				
				Even though computer vision in terms of comparison to biological vision, still remains an unsolved problem, there have still been many excellent achievements in the field to date.
			\subsection{Uses of Computer Vision}
				One of the most promenant uses of computer vision currently, is in driverless cars.
				Which in recent years, have reached a level of sophistication at which they are being approved by governments to be used on public highways \cite{web:driverlessCars}.
				Complex and expensive equipment with the capability to analyse a three-dimensional scene in real-time is required to achieve this.
				%Add the http://velodynelidar.com/lidar/hdlproducts/hdl64e.aspx here, try and find a price to reference "expensive" maybe mention google explicitely.
				Computer vision is also used on manufacturing production lines.
				\citeasnoun{journal:salmon} set out to try and classify salmon fillets, to see if it was possible to determine whether a salmon fillet had been processed using enzymes.%maybe expand on what they actually did experementally here.
				However, computer vision isn't only available to big companies with large amounts of money to spend on research.
				\begin{quote}
				``\textit{Computer vision is a rapidly growing field, partly as a result of both cheaper and more capable cameras, partly because of affordable processing power, and partly because vision algorithms are starting to mature.}''\cite[p. ix]{definition:cv}
				\end{quote}
				One such device that has been made available is the Kinect by Microsoft.
				The computer vision society found that the capabilities of the Kinect could be extended beyond its intentional use for gaming, and at a much lower cost than traditional three dimensional cameras.
				It has been used in areas such as human activity analysis, where it is able to estimate details about the pose of the human subject in its field of vision \cite{kinect:1}.
				It has also been used to for real time odometry whilst attached to a quadcopter, enabling the production of a three dimensional mapping of a physical space \cite{kinect:2}.
				%The line below is dependant on a reference
				A hacking culture has enabled this type of exploitation of technology to take place and is spurring the creators of these technologies to make them more open.
				
			\subsection{OpenCV}
			
			%Introduction to OpenCV here. Overview, not in depth.	
				OpenCV is an open source computer vision library.
				It was designed for computational efficiency with a strong focus on real-time applications.
				One of OpenCV's goals is to provide a simple-to-use computer vision infrastructure that helps people build fairly sophisticated vision applications quickly \cite[p. 1]{definition:cv}.	
				It aimed to advance computer vision research by providing open and already optimised code.
				Providing a basic vision infrastucture with no need to reinvent the wheel.
				It also provided a common infrastructure on which knowledge of computer vision could be shared between developers.
				Making code more readable and transferable \cite[p. 6]{definition:cv}. 
				This is an overview for now but more specifics will be covered later in the document.
				
		\section{The Leap Motion Controller}
			\subsection{What is the Leap Motion?}
				%cite leapmotion https://developer.leapmotion.com/articles/intro-to-motion-control
				The Leap Motion Controller is a device aimed at providing a Natural User Interface through hand gestures.
				It is made up of two infra red cameras both with a fisheye lense, set up stereoscopically.
				It provides functionality out of the box that allows a developer to track movements and gestures made by a hand \cite{web:leapGestures}.
				%Don't really need the price...
				It is discrete in size and has an accessible price. %maybe include the price here
				Leap motion have recently released a new version of their SDK, allowing access to more elements of their Leapmotion Controller. 
				Namely, images from the two cameras. 
				This allows the calculation of the disparity between the two images. %define disparity

			\subsection{Uses in Computer Vision}
		\section{Product Expectations}
		%define the boundaries of the project		
		
	%----------------------------------------- Chapter 2 -----------------------------------------%
	\chapter{Background Research}
		\section{Literature Review}
	%----------------------------------------- Chapter 3 -----------------------------------------%
	\chapter{Requirements}
	%----------------------------------------- Chapter 4 -----------------------------------------%
	\chapter{Design}
	%----------------------------------------- Chapter 5 -----------------------------------------%
	\chapter{Implementation and Test}
	%----------------------------------------- Chapter 6 -----------------------------------------%
	\chapter{Evaluation}
	%----------------------------------------- Chapter 7 -----------------------------------------%
	\chapter{Conclusion}

	\bibliographystyle{dcu}
	\bibliography{bibfile}
\end{document}
